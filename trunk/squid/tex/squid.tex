\documentclass[12pt]{article}
\usepackage[paper=a4paper,left=20mm,right=20mm,top=30mm,bottom =30mm]{geometry}
\usepackage[utf8]{inputenc}
\usepackage[T1]{fontenc}
\usepackage{stmaryrd}
\usepackage{setspace}
\usepackage{mathrsfs}
\usepackage[ngerman]{babel}
\usepackage{amssymb}
\usepackage{amsmath}
\usepackage{fancyhdr}
\usepackage[dvips,unicode,colorlinks,linkcolor=black]{hyperref} 
\usepackage{graphicx}
\usepackage{float}

\pagestyle{fancy}
\lfoot{}
\rfoot{Paul Kremser, Tobias Grussenmeyer}
\cfoot{\thepage}
\fancyhead[L]{FPI Versuch: SQUID}
\renewcommand{\headrulewidth}{0.6pt}
\renewcommand{\footrulewidth}{0.6pt}
\setlength{\headheight}{16pt}
\setlength{\parindent}{0pt}
% Für die Wahl der Schriftart
\newcommand{\changefont}[3]{
\fontfamily{#1} \fontseries{#2} \fontshape{#3} \selectfont}

\begin{document}
% keine Hurenkinder und Schusterjungen
\clubpenalty = 10000
\widowpenalty = 10000 
\displaywidowpenalty = 10000

\onehalfspacing
% Schriftart
\changefont{ptm}{m}{n} 

\begin{titlepage}
\author{Paul Kremser, Tobias Grussenmeyer}
\title{Versuch: SQUID}
\date{Versuchsdurchführung: 22. Oktober 2009} 
\maketitle
\thispagestyle{empty}
\end{titlepage}


\tableofcontents
\thispagestyle{empty}
\newpage
\pagenumbering{arabic}
\section{Überblick}
In diesem Versuch soll mittels eines \textit{SQUID} das Magnetfeld verschiedener Proben sowie einer Stromdurchflossenen Leiterschleife gemessen und deren Dipolmoment bestimmt werden. Ein SQUID (\textbf{S}uper\-conducting \textbf{Qu}antum \textbf{I}nterference \textbf{D}evice) ist ein sehr kleiner, höchst empfindlicher Magnetfelddetektor, mit dem sich äußerst geringe Magnetfeldschwankungen registrieren lassen. Da die Nachweisgenauigkeit nur durch Quanteneffekte begrenzt ist, wird die Messgenauigkeit von keinem anderen Magnetfelddetektor übertroffen. Im wesentlichen besteht ein SQUID nur aus einem supraleitendem Ring der an einer sehr schmalen Stelle durch einen Isolator unterbrochen ist. Mit Hilfe geeigneter Elektronik können so kleinste Flussänderungen im Innern des Rings erkannt werden.
\section{Aufgabenstellung}
\begin{enumerate}
 \item Kalibration das SQUID durch Variation der Einstellungen. Es soll eine möglichst große Amplitude des SQUID-Patterns bei möglichst kleinem Rauschen gefunden werden.
\item Vermessung der Magnetfelder einer rotierender Leiterschleife die von fünf verschiedenen Stromstärken durchflossen wird. Bestimmung der Dipolmomente und Vergleich mit berechnetem Wert.
\item Vermessung der Magnetfelder und Diplomomente verschiedener Proben (Magnetspan, Stabmagnet, Geldstück, SIM-Karte)
\item (optional) polare Darstellung der Stärke des Magnetfeldes in Abhängigkeit des Drehwinkels
\end{enumerate}
\newpage

\section{Theoretische Grundlagen}

\section{Versuchsaufbau}

\section{Durchführung}

\section{Auswertung}

\section{Zusammenfassung}

\end{document}
