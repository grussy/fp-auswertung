\documentclass[12pt]{article}
\usepackage[paper=a4paper,left=20mm,right=20mm,top=30mm,bottom =30mm]{geometry}
\usepackage[utf8]{inputenc}
\usepackage[T1]{fontenc}
\usepackage{stmaryrd}
\usepackage{setspace}
\usepackage{mathrsfs}
\usepackage[ngerman]{babel}
\usepackage{amssymb}
\usepackage{amsmath}
\usepackage{fancyhdr}
\usepackage[dvips,unicode,colorlinks,linkcolor=black]{hyperref} 
\usepackage{graphicx}
\usepackage{float}

\pagestyle{fancy}
\lfoot{}
\rfoot{Paul Kremser, Tobias Grussenmeyer}
\cfoot{\thepage}
\fancyhead[L]{FPI Versuch: SQUID}
\renewcommand{\headrulewidth}{0.6pt}
\renewcommand{\footrulewidth}{0.6pt}
\setlength{\headheight}{16pt}
\setlength{\parindent}{0pt}
% Für die Wahl der Schriftart
\newcommand{\changefont}[3]{
\fontfamily{#1} \fontseries{#2} \fontshape{#3} \selectfont}

\begin{document}
% keine Hurenkinder und Schusterjungen
\clubpenalty = 10000
\widowpenalty = 10000 
\displaywidowpenalty = 10000

\onehalfspacing
% Schriftart
\changefont{ptm}{m}{n} 

\begin{titlepage}
\author{Paul Kremser, Tobias Grussenmeyer}
\title{Versuch: SQUID}
\date{Versuchsdurchführung: 22. Oktober 2009} 
\maketitle
\thispagestyle{empty}
\end{titlepage}


\tableofcontents
\thispagestyle{empty}
\newpage
\pagenumbering{arabic}
\section{Überblick}
In diesem Versuch soll mittels eines \textit{SQUID} das Magnetfeld verschiedener Proben sowie einer Stromdurchflossenen Leiterschleife gemessen und deren Dipolmoment bestimmt werden. Ein SQUID (\textbf{S}uper\-conducting \textbf{Qu}antum \textbf{I}nterference \textbf{D}evice) ist ein sehr kleiner, höchst empfindlicher Magnetfelddetektor, mit dem sich äußerst geringe Magnetfeldschwankungen registrieren lassen. Da die Nachweisgenauigkeit nur durch Quanteneffekte begrenzt ist, wird die Messgenauigkeit von keinem anderen Magnetfelddetektor übertroffen. Im wesentlichen besteht ein SQUID nur aus einem supraleitendem Ring der an einer sehr schmalen Stelle durch einen Isolator unterbrochen ist. Mit Hilfe geeigneter Elektronik können so kleinste Flussänderungen im Innern des Rings erkannt werden.
\section{Aufgabenstellung}
\begin{enumerate}
 \item Kalibration das SQUID durch Variation der Einstellungen. Es soll eine möglichst große Amplitude des SQUID-Patterns bei möglichst kleinem Rauschen gefunden werden.
\item Vermessung der Magnetfelder einer rotierender Leiterschleife die von fünf verschiedenen Stromstärken durchflossen wird. Bestimmung der Dipolmomente und Vergleich mit berechnetem Wert.
\item Vermessung der Magnetfelder und Diplomomente verschiedener Proben (Magnetspan, Stabmagnet, Geldstück, SIM-Karte)
\item (optional) polare Darstellung der Stärke des Magnetfeldes in Abhängigkeit des Drehwinkels
\end{enumerate}
\newpage

\section{Theoretische Grundlagen}
\subsection{Supraleitung}
Unter Supraleitung versteht man elektrische Leitung ohne messbaren Widerstand, man spricht auch von einem Suprastrom.
Es sind zwei Arten von Supraleitern bekannt: Typ-I und Typ-II. \\

Bei Typ-I-Supraleitern wird die Supraleitung durch eine Paarbildung von Elektronen (Cooper-Paare) im Leiter erklärt.
Metalle werden als Kristalle betrachtet, in denen sich Leitungselektronen praktisch frei zwischen den Atomen bewegen können. Dieses „Elektronengas“ besteht aus Fermionen und unterliegt deshalb der Fermi-Verteilung welche die Besetzung der Energiezustände angibt. Wird das Material gekühlt, verringert sich die Atombewegung. Bei ausreichend tiefen Temperaturen bewegen sich die Elektronen so langsam, dass ein Elektron aufgrund seiner Ladung Gitteratome anzieht. Die Bewegung der angezogenen Atome zieht sich als Welle durch das ganze Medium und wird als Phonon bezeichnet. Die Gegenbewegung der Atome erfolgt aufgrund ihrer höheren Masse zeitlich stark verzögert und daraus resultiert eine Polarisation des Gitters, die die Coulombabstoßung überkompensiert. Ein zweites Elektron kann nun in dieser Polarisationsspur seine Energie absenken, d.h. es wird gebunden. Es entsteht, vermittelt über die Gitterbewegung, ein Cooper-Paar.\\

Da sich die beiden beteiligten Elektronen in entgegengesetzter Richtung bewegen, ist der Gesamtimpuls annährend Null.
Mit der Unschärferelation folgt somit eine sehr große Ortsausdehnung. Ein Cooper-Paar hat eine Ausdehnung in der Größenordnung von 1000\AA. Die Wellenfunktionen der einzelnen Cooper-Paare überlappen also mehrfach, wodurch alle Cooper-Paare durch eine gemeinsame Gesamtwellenfunktion beschrieben werden kann.\\

Bei der normalen elektrischen Leitung entsteht der elektrische Widerstand durch Wechselwirkungen der Elektronen mit Gitterfehlern des Kristallgitters und Gitterschwingungen. Darüber hinaus können auch Streuprozesse der Elektronen untereinander eine wichtige Rolle spielen. Durch die Kopplung der Elektronen im Supraleiter zu Cooper-Paaren wird die Energieabgabe an das Kristallgitter unterdrückt und so der widerstandslose elektrische Stromfluss ermöglicht. Die beiden einzelnen Elektronen sind Fermionen, die sich zu einem bosonischen Cooper-Paar zusammenschließen. (Siehe dazu \ref{bcs})\\

Supraleiter 2. Art befinden sich nur bis zu einem kritischen äußeren Magnetfeld in der Meißner-Phase, verhalten sich also wie Typ I, darüber können magnetische Feldlinien in Form so genannter Flussschläuche in das Material eindringen, ehe der supraleitende Zustand bei einem oberen kritischen Magnetfeld vollständig zerstört wird. Der magnetische Fluss in den Flussschläuchen beträgt immer ein ganzzahliges Vielfaches des magnetischen Flussquants (siehe \ref{flussquantisierung})

Beispiele für Typ-II-Supraleiter sind die keramischen Hochtemperatursupraleiter. Zwei wichtige Gruppen sind YBaCuO (Yttrium-Barium-Kupferoxide) und BiSrCaCuO (Bismut- Strontium- Kalzium- Kupferoxide). Weiterhin zählen die meisten supraleitenden Legierungen zum Typ II, so die für MR-Magnete verwendeten Niob-Aluminium-Legierungen.

\subsection{BCS-Theorie}
\label{bcs}
Ein erstes Elektron verändert das Gitter durch Energieabgabe (Phonon) derart, dass ein zweites Elektron (z.B. durch Veränderung seiner Bahn oder Aufnahme eines Phonons) einen gleichgroßen Energiegewinn erzielt. Dies ist nur möglich, falls die Gitterbausteine und die Elektronen sich langsam genug (geringe  Temperatur, Stromdichte) bewegen.\\

Die Idee der BCS-Theorie besteht darin, die Bildung sogenannter Cooper-Paare aus je zwei Elektronen durch eine schwache anziehende Wechselwirkung zu postulieren. Elektronen sind aufgrund ihres Spins (se=1/2) Fermionen und können als solche nicht den gleichen Zustand besetzen (Pauli-Prinzip). Im Gegensatz dazu sind die Cooper-Paare mit Spin s=0 (antiparallele Anordnung der Elektronenspins) Bosonen und können daher gleichzeitig den gleichen Zustand, und somit auch alle den Grundzustand annehmen. Dies ist nicht nur energetisch günstiger, sondern äußert sich auch in einer, den ganzen Festkörper überspannenden, Bose-Einstein-Wellenfunktion.\\

Diese Wellenfunktion kann von lokalen Hindernissen wie Atomkernen und Störstellen des Gitters allgemein nicht mehr beeinflusst werden und garantiert somit einen widerstandslosen Ladungstransport.

\subsection{Flussquantisierung}
\label{flussquantisierung}


\section{Versuchsaufbau}

\section{Durchführung}

\section{Auswertung}

\section{Zusammenfassung}

\end{document}
