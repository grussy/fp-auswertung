\documentclass[12pt]{article}
\usepackage[paper=a4paper,left=30mm,right=30mm,top=35mm,bottom =35mm]{geometry}
\usepackage[utf8]{inputenc}
\usepackage[T1]{fontenc}
\usepackage{stmaryrd}
\usepackage{setspace}
\usepackage{mathrsfs}
\usepackage[ngerman]{babel}
\usepackage{amssymb}
\usepackage{amsmath}
\usepackage{fancyhdr}
\usepackage[dvips,unicode,colorlinks,linkcolor=black]{hyperref} 
\usepackage{graphicx}
\usepackage{float}

\pagestyle{fancy}
\lfoot{}
\rfoot{Paul Kremser, Tobias Grussenmeyer}
\cfoot{\thepage}
\fancyhead[L]{FPII Versuch: Holographie}
\renewcommand{\headrulewidth}{0.6pt}
\renewcommand{\footrulewidth}{0.6pt}
\setlength{\headheight}{30pt}
\setlength{\parindent}{0pt}
% Für die Wahl der Schriftart
\newcommand{\changefont}[3]{
\fontfamily{#1} \fontseries{#2} \fontshape{#3} \selectfont}
%\renewcommand{\vec}[1]{\mathbf{#1}}
\renewcommand*\vec[1]{{\mbox{\boldmath\ensuremath{#1}}}}

\begin{document}
% keine Hurenkinder und Schusterjungen
\clubpenalty = 10000
\widowpenalty = 10000 
\displaywidowpenalty = 10000

\onehalfspacing
% Schriftart
\changefont{ptm}{m}{n} 

\begin{titlepage}
\author{Paul Kremser, Tobias Grussenmeyer}
\title{Versuch: Holographie}
\date{Versuchsdurchführung: 15. bis 19. März 2010} 
\maketitle
\thispagestyle{empty}
\end{titlepage}


\tableofcontents
\thispagestyle{empty}
\newpage
\pagenumbering{arabic}
\section{Überblick}
Die Holographie ist ein Verfahren zur dreidimensionalen Abbildung von
Gegenständen. Zudem existiern auch einige sehr interessante Messmethoden die auf der Holographie basieren. Beispielsweise ist es durch Überlagerung von 
Hologramm und wahrem Objekt möglich kleinste Veränderungen am Objekt festzustellen. 

\section{Aufgabenstellung}
\begin{itemize}
 \item Mit einem Michelson-Interferometer soll beobachtet werden wie der Optische Aufbau auf Störungen von Außen reagiert und die Kohärenzlänge des Lasers 
      bestimmt werden.
 \item Anfertigung eines Doppelbelichtungshologramms zur Feststellung der Durchbiegung der drei Balken
 \item Untersuchung der Eigenschwingungen der Aluminiumplatte mittels Echtzeitholographie
 \item Beobachtung der Kreuzkorrelation zweier gegeneinander verdrehter Spalte mittels Fourierspektroskopie
\end{itemize}


\section{Theoretische Grundlagen}
\subsection{Michelson-Interferometer}
\subsection{Hologramme}
Auf normalen Fotos ist auf dem Film lediglich Information über die Intensität und bei Farbfotos die Farbe des Lichtes enthalten.

Bei der Holografie wird auf dem Film die Phase und die Intensität gespeichert. Möglich ist dieses durch die Interferenz zweier kohärenter Lichtstrahlen.
Dem sogenannten Referenzstrahl und Objektstrahl
\subsubsection{Aufnahme}
\subsubsection{Rekonstruktion}

\section{Versuchsaufbau}

\section{Durchführung}
\section{Auswertung}
\section{Zusammenfassung}

\end{document}
