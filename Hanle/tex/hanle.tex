\documentclass[12pt]{article}
\usepackage[paper=a4paper,left=20mm,right=20mm,top=30mm,bottom =30mm]{geometry}
\usepackage[utf8]{inputenc}
\usepackage[T1]{fontenc}
\usepackage{stmaryrd}
\usepackage{setspace}
\usepackage{mathrsfs}
\usepackage[ngerman]{babel}
\usepackage{amssymb}
\usepackage{amsmath}
\usepackage{fancyhdr}
\usepackage[dvips,unicode,colorlinks,linkcolor=black]{hyperref} 
\usepackage{graphicx}
\usepackage{float}

\pagestyle{fancy}
\lfoot{}
\rfoot{Paul Kremser, Tobias Grussenmeyer}
\cfoot{\thepage}
\fancyhead[L]{FPI Versuch: Hanle - Effekt}
\renewcommand{\headrulewidth}{0.6pt}
\renewcommand{\footrulewidth}{0.6pt}
\setlength{\headheight}{16pt}
\setlength{\parindent}{0pt}
% Für die Wahl der Schriftart
\newcommand{\changefont}[3]{
\fontfamily{#1} \fontseries{#2} \fontshape{#3} \selectfont}

\begin{document}
% keine Hurenkinder und Schusterjungen
\clubpenalty = 10000
\widowpenalty = 10000 
\displaywidowpenalty = 10000

\onehalfspacing
% Schriftart
\changefont{ptm}{m}{n} 

\begin{titlepage}
\author{Paul Kremser, Tobias Grussenmeyer}
\title{Versuch: Hanle - Effekt}
\date{Versuchsdurchführung: 9. und 12. Oktober 2009} 
\maketitle
\thispagestyle{empty}
\end{titlepage}


\tableofcontents
\thispagestyle{empty}
\newpage
\pagenumbering{arabic}
\section{Überblick}
Mit dem Versuch soll die Lebensdauer angeregter Zustände des Queckilberatoms bestimmt werden. Grob gesagt werden hierzu Quecksilberatome in einer 
Quarzzelle durch polarisiertes Licht angeregt. Durch ein von außen angelegtes Magnetfeld wird die Richtung
in der die angeregten Atome wieder abstrahlen beeinflusst.

\section{Aufgabestellung}
Bestimmung der Lebensdauer des angeregten $6s6p \quad ^3P_1$ Zustands des $Hg$-Atoms durch Messung des Hanle-Signals in Abhängigkeit vom Dampfdruck
des Quecksilbers für verschiedene Polarisationsrichtungen des einstrahlenden Lichts. Dokumentation der Änderung der effektiven Lebensdauer durch
\textit{Coherence Narrowing}

\section{Theoretische Grundlagen}

\section{Versuchsaufbau}

\section{Durchführung}

\section{Auswertung}

\section{Zusammenfassung}

\end{document}
