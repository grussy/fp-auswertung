\documentclass[12pt]{article}
\usepackage[paper=a4paper,left=20mm,right=20mm,top=30mm,bottom =30mm]{geometry}
\usepackage[utf8]{inputenc}
\usepackage[T1]{fontenc}
\usepackage{stmaryrd}
\usepackage{setspace}
\usepackage{mathrsfs}
\usepackage[ngerman]{babel}
\usepackage{amssymb}
\usepackage{amsmath}
\usepackage{fancyhdr}
\usepackage[dvips,unicode,colorlinks,linkcolor=black]{hyperref} 
\usepackage{graphicx}
\usepackage{float}

\pagestyle{fancy}
\lfoot{}
\rfoot{Paul Kremser, Tobias Grussenmeyer}
\cfoot{\thepage}
\fancyhead[L]{FPI Versuch: Ultraschall}
\renewcommand{\headrulewidth}{0.6pt}
\renewcommand{\footrulewidth}{0.6pt}
\setlength{\headheight}{16pt}
\setlength{\parindent}{0pt}
% Für die Wahl der Schriftart
\newcommand{\changefont}[3]{
\fontfamily{#1} \fontseries{#2} \fontshape{#3} \selectfont}

\begin{document}
% keine Hurenkinder und Schusterjungen
\clubpenalty = 10000
\widowpenalty = 10000 
\displaywidowpenalty = 10000

\onehalfspacing
% Schriftart
\changefont{ptm}{m}{n} 

\begin{titlepage}
\author{Paul Kremser, Tobias Grussenmeyer}
\title{Versuch: Ultraschall}
\date{Versuchsdurchführung: 6. und 7. Oktober 2009} 
\maketitle
\thispagestyle{empty}
\end{titlepage}


\tableofcontents
\thispagestyle{empty}
\newpage
\pagenumbering{arabic}
\section{Überblick}

\section{Aufgabestellung}
\textbf{Amplitudengitter}
\begin{itemize}
 \item Bestimmung der Gitterkonstante eines Sinusgitters aus dem Abstand des 1. Beugungsordnung
 \item Bestimmung der Gitterkonstanten von 5 Amplitudengittern
 \item Berechnung der Aperturfunktion für Gitter Nr. 1 (größte Gitterkonstante, höchste Dichte an Beugungsmaxima) aus den ermittelten Intensitäten
 der Beugungsordnungen und Zeichnen einer Periode der Aperturfunktion
 \item Bestimmung des Verhältnisses der Spaltbreite zum Spaltabstand aus der Aperturfunktion
 \item Bestimmung des Auflösevermögens der Gitter bei ihrer vollen Ausleuchtung
\end{itemize}

\textbf{Phasengitter}
\begin{itemize}
 \item Messung der Intensitätsverteilung der Beugungsfigur eines Ultraschallwellengitters (Phasengitter) in Abhängigkeit von der Spannung am
 Ultraschallschwingquarz
 \item Vergleich der Messergebnisse mit der Raman-Nath-Theorie
 \item Bestimmung der Schallwellenlänge in Isooktan durch Ausmessen der Beugungsordnungen un Vergleich mit dem rechnerischen Wert
\end{itemize}



\section{Theoretische Grundlagen}

\subsection{Beugung}
Unter Beugung versteht man eine Richtungsänderung elektromagnetischer Wellen an einem Hinderniss welche nicht von Reflexion oder Brechung herrührt. Dabei unterscheidet man zwischen zwei unterschiedlichen Versuchsanordnungen:
\begin{itemize}
 \item Fresnel’sche Anordnung: Bei dieser Anordnung ist der Abstand Lichtquelle-Apertur oder Schirm - Apertur klein, d.h. die Form der Wellenfront des einfallenden Lichts ist nicht vernachlässigbar.
 \item Frauenhofer Anordnung: Im Gegensatz zur Fresnelschen Anordnung sind hier die Abstände groß, d.h. die Wellenfront des einfallenden Lichts kann als eben angenommen werden. Zusätzlich wird gefordert, daß die Beugungsöffnung groß gegen die Wellenlänge des einfallenden Lichtes ist.
\end{itemize}
Hier wird Beugung an Amplituden und Phasengittern an einer Frauenhofer Anordnung betrachtet.

\subsection{Amplitudengitter}
Jedes Gitter wird durch seine Gitterkonstante $K$ charakterisiert. Die Gitterkonstante gibt den Abstand zweier benachbarter Spaltmitten an.
Bei bekannter Wellenlänge $\lambda$ lässt sich die Gitterkonstante aus dem Winkel des $m$-ten Intensitätsmaximum gegenüber dem 0-ten Maximum berechnen:
\begin{align}
 K = \frac{m \ \lambda}{\sin{\theta}}
\end{align}

\subsection{Aperturfunktion}
Jedem Objekt, an dem Beugung stattfindet, lässt sich eine Aperturfunktion $g$ zuordnen. Die Aperturfunktion beschreibt die Eigenschaften des Objekts indem
sie jedem Punkt in der Blendenebene einen Wert zuorndet. Mit Hilfe des Kirchhoffschen Integraltheorems und der daraus erhaltenen Integralformel für die
Amplitude einer Kugelwelle auf dem Rand ihres Ausbreitungsgebietes kann man zeigen, dass die Fouriertransformierte der Aperturfunktion $g$ des
Beugungsobjekts gerade die Intensitätsverteilung $I$ des Beugungsbildes ist.
\begin{align}
 I = \lvert \Psi(x,y) \rvert^2 = \left\lvert \int_{Blende}{g \ e^{-ikr} \ dA} \right\rvert
\end{align}

Umgekehrt lässt sich durch Fouriertransformation aber auch die Aperturfunktion aus der Intensitätsverteilung gewinnen.
Die Fouriertransformierte einer Funktion ist wie folgt definiert:
\begin{align}
 F(g) = \int\limits_{-\infty}^{\infty}g(x)~e^{-ikx}~dx
\end{align}

Ist die genaue Intensitätsverteilung nicht vollständig bekannt so kann man die Aperturfunktion ersatzweise in einer Fourierreihe nähern.
Hierzu werden die Wurzeln der Intensitätsmaxima als Koeffizienten genommen:
\begin{align}
 g(x) = \sum_{j=0}^{\infty}{\pm \sqrt{I_j} \ \cos \left( \frac{x}{K} \ 2 \ \pi \ j \right) }
\end{align}

Bei einem Sinusgitter (hier treten nur Maxima 0-ter und 1-ter Ordnung auf) ergibt sich z.B.:
\begin{align*}
 g(x) = \sqrt{I_0} \ + \ \sqrt{I_1} \ \cos\left( \frac{x}{K} \ 2 \ \pi \right)
\end{align*}

Aus dem erhaltenen Graph die Aperturfunktion kann man dann die Spaltbreite und den Spaltabstand ablessen.

\subsection{Auflösungsvermögen}
Die Definition des Auflösungsvermögen $a$ ist:
\begin{align}
 a = \frac{\lambda}{\Delta\lambda}
\end{align}

mit der Wellenlänge $\lambda$ und dem Wellenlängenabstand $\Delta\lambda$ bei dem eine von $\lambda$ verschiedene Wellenlänge bei Beugung gerade noch
unterscheiden lässt. Zudem gilt:
\begin{align}
 a = N \ m
\end{align}
mit der Anzahl der augeleuchteten Gitterlinien $N$ und der größten beobachteten Beugungsordnung $m$.

\subsection{Phasengitter}
%VORHER: Die Beugung am Phasengitter ist in unserem Fall vernachlässigbar. Stattdessen wird der Strahl am Phasengitter unterschiedlich stark gebrochen.
Hier ist - im Gegensatz zum Amplitudengitter - die Transmission konstant, moduliert wird die Phase.
Eine Ultraschallwelle erzeugt in Isooktan Bereiche unterschiedlicher Dichte und somit unterschiedlicher Brechungsindizes.
Somit treten ehmals gleichphasige Wellen phasenverschoben aus dem Gitter aus und können miteinander interferieren. Für den brechungsindex $n$ in der 
Flüssigkeit gilt:
\begin{align}
 n(x) = n_0 + \Delta n \ sin\left(\frac{2\ \pi}{\Lambda}\ x\right)
\end{align}
mit $\Lambda$ der Schallwellenlänge, welche das Analogon dur Gitterkonstante beim Amplitudengitter ist.
Da $\Delta n$ proportional zur Schallintensität $S$ ist können wir den Brechungsindex mit der Amplitude der Schallwelle ändern.

\subsection{Raman-Nath-Theorie}
Für die Winkel der Intensitätsmaxima des Beugungsbildes gilt in $m$-ter Ordnung:
\begin{align}
 \sin~\theta = \pm~\frac{\lambda}{\Lambda}~m
\end{align}

Die Intensitäten der Maxima $m$-ter Ordnung verhalten sich zu denen $m'$-ter Ordnung wie die Besselfunktion der $m$-ten zu denen der $m'$-ten Ordnung.
\begin{align}
 \frac{I_m}{I_{m'}} = \frac{J^2_m(\Delta n~D \times 2~\pi~/~\lambda)}{J^2_{m'}(\Delta n \times 2~\pi~/~\lambda)}
\end{align}






\section{Versuchsaufbau}

\section{Durchführung}

\section{Auswertung}

\section{Zusammenfassung}

\end{document}
