\documentclass[12pt]{article}
\usepackage[paper=a4paper,left=30mm,right=30mm,top=35mm,bottom =35mm]{geometry}
\usepackage[utf8]{inputenc}
\usepackage[T1]{fontenc}
\usepackage{stmaryrd}
\usepackage{setspace}
\usepackage{mathrsfs}
\usepackage[ngerman]{babel}
\usepackage{amssymb}
\usepackage{amsmath}
\usepackage{fancyhdr}
\usepackage[dvips,unicode,colorlinks,linkcolor=black]{hyperref} 
\usepackage{graphicx}
\usepackage{float}

\pagestyle{fancy}
\lfoot{}
\rfoot{Paul Kremser, Tobias Grussenmeyer}
\cfoot{\thepage}
\fancyhead[L]{FPII Versuch: Holographie}
\renewcommand{\headrulewidth}{0.6pt}
\renewcommand{\footrulewidth}{0.6pt}
\setlength{\headheight}{30pt}
\setlength{\parindent}{0pt}
% Für die Wahl der Schriftart
\newcommand{\changefont}[3]{
\fontfamily{#1} \fontseries{#2} \fontshape{#3} \selectfont}
%\renewcommand{\vec}[1]{\mathbf{#1}}
\renewcommand*\vec[1]{{\mbox{\boldmath\ensuremath{#1}}}}

\begin{document}
% keine Hurenkinder und Schusterjungen
\clubpenalty = 10000
\widowpenalty = 10000 
\displaywidowpenalty = 10000

\onehalfspacing
% Schriftart
\changefont{ptm}{m}{n} 

\begin{titlepage}
\author{Paul Kremser, Tobias Grussenmeyer}
\title{Versuch: Holographie}
\date{Versuchsdurchführung: 15. bis 19. März 2010} 
\maketitle
\thispagestyle{empty}
\end{titlepage}


\tableofcontents
\thispagestyle{empty}
\newpage
\pagenumbering{arabic}
\section{Überblick}
Die Holographie ist ein Verfahren zur dreidimensionalen Abbildung von
Gegenständen. Zudem existiern auch einige sehr interessante Messmethoden die auf der Holographie basieren. Beispielsweise ist es durch Überlagerung von 
Hologramm und wahrem Objekt möglich kleinste Veränderungen am Objekt festzustellen. 

\section{Aufgabenstellung}
\begin{itemize}
 \item Mit einem Michelson-Interferometer soll beobachtet werden wie der Optische Aufbau auf Störungen von Außen reagiert und die Kohärenzlänge des Lasers 
      bestimmt werden.
 \item Anfertigung eines Doppelbelichtungshologramms zur Feststellung der Durchbiegung der drei Balken
 \item Untersuchung der Eigenschwingungen der Aluminiumplatte mittels Echtzeitholographie
 \item Beobachtung der Kreuzkorrelation zweier gegeneinander verdrehter Spalte mittels Fourierspektroskopie
\end{itemize}


\section{Theoretische Grundlagen}
\subsection{Michelson-Interferometer}
\subsection{Hologramme}
Auf normalen Fotos ist auf dem Film lediglich Information über die Intensität und bei Farbfotos die Farbe des Lichtes enthalten.

Bei der Holografie wird auf dem Film die Phase und die Intensität gespeichert. Möglich ist dieses durch die Interferenz zweier kohärenter Lichtstrahlen.
Dem sogenannten Referenzstrahl und Objektstrahl. Hierzu wird ein Laser verwedet der durch geeignete Optik aufgefächert wird.
\subsubsection{Aufnahme}
Wird ein beliebiges Objekt mit kohärentem Lich beleuchtet so wird das Licht gebrochen und gestreut. Es entsteht ein Wellenfeld, welches man mit dem Auge
beobachten kann. Dieses Wellenfeld wird Objektwelle genannt. Diese Welle wird mit dem einfallenden ungestreuten Licht (der so genannten Referenzwelle)
überlagert. Das entstehenden Interferenzmuster wird auf einem Lichtempfindlichen Film aufgezeichnet. Die photoempfindliche Schicht des Films reagiert nur auf
die Intensität des Lichtes, durch das Interferenzmuster wird aber die relative Phase (zwischen Objekt- und Referenzwelle) aufgezeichnet.

Voraussetzungen für die Aufzeichnung von Hologrammen ist die zeitliche und räumliche Stabilität der Interferenzmuster. Die aufzuzeichnenden Objekte dürfen
sich während der minutenlangen Belichtung nicht bewegen. Deswegen befindet sich der Aufbau zur Aufnahme des Hologramms auf einem scheren Tisch welcher auf 
einem dämpfenden Material gelagert ist. 

Nach der Entwicklung des Films erhält man ein Gitter, je nach Art des Films bzw. der Entwicklung ist dies ein Amplitudengitter (partikel im Film absorbieren
Teile des Lichts) oder ein Phasengitter (unterschiedliche Brechzahlen im Film), mit dem das Objekt rekonstruiert werden kann.

\subsubsection{Rekonstruktion}
Bei der Rekonstruktion beleuchtet man den Film mit der Referenzwelle. Dabei wird das Licht am Interferenzmuster gebeugt und es entsteht die exakte
Wellenfront der Objektwelle. Hinter dem Hologramm sieht man also den abgebildeten Gegenstand wie durch ein Fenster.
Da auch das ganze Wellenfeld vor und hinter dem aufgezeichneten Objekt rekonstruiert wird, ist man in der Lage das Objekt dreidimensional zu sehen. Dies
wird dadurch weiter verstärkt, dass man sich sogar im Wellenfeld hin- und herbewegen und so den Gegenstand aus verschiedenen Richtungen und, in begrenztem
Ausmaß, auch um ihn herum sehen kann.

Jeder Punkt des abgebildeten Objektes beeinflusst das Wellenmuster des gesamten holografischen Bildträgers. Wenn also ein Hologramm zerteilt wird, kommt bei
der Rekonstruktion noch immer das ganze Bild zustande. Das Aufteilen des Hologramms in einzelne Stücke führt lediglich zu einer Verschlechterung der
Auflösung des Bildes und zu einer Verringerung des ansehbaren räumlichen Bildwinkels.

\subsubsection{Transmissionshologramme}
Transmissionshologramme befinden sich auf lichtdurchlässigen Filmen. Bei der Aufnahme treffen Referenzstrahl und Objektstrahl von der gleichen Seite auf
den Film und erzeugen dort das Interferenzmuster, das aufgenommen wird.
Für die Rekonstruktion des Bildes muss das Transmissionshologramm von hinten mit einer kohärenten Lichtquelle durchleuchtet werden.

\subsubsection{Reflexionshologramme}
Reflexionshologramme reflektieren das einfallende Licht, so dass die Lichtquelle im Gegensatz zu Transmissionshologrammen auf der Seite des Betrachters sein
kann. Bei Reflexionshologrammen treffen Objektstrahl und Referenzstrahl von unterschiedlichen Seiten auf den Film und bilden in ihm das Interferenzmuster, das 
den Film belichtet. Reflexionshologramme sind in jedem Fall Volumenhologramme, d.h. die Dicke des Aufnahmematerials wird zur Speicherung des holografischen 
Bilds genutzt. Es entstehen in dem Film verschiedene Netzebenen, die durch das an den Interferenzmaxima belichteten Stellen des Filmmaterials gehen. Die 
Netzebenen reflektieren bei der Rekonstruktion des Hologramms das einfallende Licht so zurück, dass ein Bild des Gegenstands entsteht. Reflexionshologramme 
sind wegen der Bragg-Bedingung Weißlichthologramme d.h. sie können mit weißem Licht rekonstruiert werden. Hierbei ändert sich die Farbe des virtuellen Bildes
je nach einfallswikel des Lichtes.

\subsection{Doppelbelichtungsholographie}
Bei der Doppelbelichtungsholographie werden zwei verschiedene (räumliche) Zustände des Objekts nacheinander auf einen Film belichtet. Für jeden Zustand wird
dabei die halbe Belichtungszeit verwendet. Ist der Unterschied der beiden Zustände nicht zu groß so lässt sich bei der Rekonstruktion eine Überlagerung der
beiden Zustände beobachten. In diesem Versuch werden drei an einem Ende eingespannte Metallbalken durch kleine Gewichte verbogen. Später lassen sich auf dem
virtuellen Bild der Balken Interferenzstreifen beobachten. Aus dem Abstand dieser Streifen kann auf die Stärke der Durchbiegung zurückgeschlossen werden.
Die Durchbiegungskurven haben die Form:
\begin{align}
 y = P_1 \left(\frac{5 x^2 - x^3}{6}\right) + P_2 x + P_3
\end{align}
wobei $x$ der Abstand von der Einspannstelle, $y$ die Auslenkung aus der Ruhelage und $P_{1,2,3}$ Material bzw. Aufbaubedingt sind.

\section{Durchführung}
\section{Auswertung}
\section{Zusammenfassung}

\end{document}
