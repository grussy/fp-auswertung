\documentclass[12pt]{article}
\usepackage[paper=a4paper,left=30mm,right=30mm,top=35mm,bottom =35mm]{geometry}
\usepackage[utf8]{inputenc}
\usepackage[T1]{fontenc}
\usepackage{stmaryrd}
\usepackage{setspace}
\usepackage{mathrsfs}
\usepackage[ngerman]{babel}
\usepackage{amssymb}
\usepackage{amsmath}
\usepackage{fancyhdr}
\usepackage[dvips,unicode,colorlinks,linkcolor=black]{hyperref} 
\usepackage{graphicx}
\usepackage{float}
\usepackage{listings}

\pagestyle{fancy}
\lfoot{}
\rfoot{Paul Kremser, Tobias Grussenmeyer}
\cfoot{\thepage}
\fancyhead[L]{FPII Versuch: Laserspektroskopie}
\renewcommand{\headrulewidth}{0.6pt}
\renewcommand{\footrulewidth}{0.6pt}
\setlength{\headheight}{30pt}
\setlength{\parindent}{0pt}
% Für die Wahl der Schriftart
\newcommand{\changefont}[3]{
\fontfamily{#1} \fontseries{#2} \fontshape{#3} \selectfont}
%\renewcommand{\vec}[1]{\mathbf{#1}}
\renewcommand*\vec[1]{{\mbox{\boldmath\ensuremath{#1}}}}

\lstset{
	basicstyle=\ttfamily\scriptsize\mdseries,
	keywordstyle=\bfseries\color{black},
	identifierstyle=,
	commentstyle=\color{black},	
	stringstyle=\itshape\color{black},
	numbers=left,
	numberstyle=\tiny,
	stepnumber=1,
	breaklines=true,
	frame=none,
	showstringspaces=false,
	tabsize=4,
	backgroundcolor=\color{white},
	captionpos=b,
	float=htbp,
	frame=tlrb
%	extendedchars=true
}

\begin{document}
% keine Hurenkinder und Schusterjungen
\clubpenalty = 10000
\widowpenalty = 10000 
\displaywidowpenalty = 10000

\onehalfspacing
% Schriftart
\changefont{ptm}{m}{n} 

\begin{titlepage}
\author{Paul Kremser, Tobias Grussenmeyer}
\title{Versuch: Laserspektroskopie}
\date{Versuchsdurchführung: 30. März bis 2.April 2010} 
\maketitle
\thispagestyle{empty}
\end{titlepage}
%Leider schafften wir nur einen geringen Teil der gestellten Aufgaben. Ein wichtiger Teil unserer Messdataen, der des 
%letzten Tages unserer Messungen, ging verloren. Aufgrund eines Feiertages hatten wir desweiteren nur vier von fünf Möglichen Tagen zur Verfügung.

\tableofcontents
\thispagestyle{empty}
\newpage
\pagenumbering{arabic}
\section{Überblick}
In diesem Versuch soll das Prinzip der Laserspektroskopie verstanden und an einem Beispiel angewand werden. Zusätzlich sollen Messungen zur Polarisation von
Licht und Magnetooptik gemacht werden.
Unter Laserspektroskopie versteht man verschiedene Verfahren zur Spektroskopie, in denen Laser zur Untersuchung von atomaren oder molekularen Spektren eingesetzt werden. 
In unserem Versuch wird das zu untersuchende Atom ($^{85}Rb$ und $^{87}Rb$) durch Lasereinstrahlung von einem unteren auf ein höheres Energienievau gehoben.
Dadurch kommt es beim variieren der Laserfrequenz zu Absorbtionsmaxima die mit einem Detektor gemessen werden können.
\newpage
\section{Aufgabenstellung}
\begin{itemize}
 \item Der Laser
  \begin{itemize}
   \item Bestimme die Laserschwelle 
   \item Nimm für drei Wiederstände eine Spannungskurve auf
  \end{itemize}
 \item Resonator und Frequenzmodulation
  \begin{itemize}
   \item Bestimme den freien Spektralbereich $\Delta f_{FSR}$ des Resonators
   \item Aus $\Delta f_{FSR}$ die Länge des Resonators
   \item Die Finesse F des Resonators
   \item Miss die Strom-Frequenz Charakteristik der Laserdiode
   \item Charakterisiere die Frequenzmodulation
  \end{itemize}
 \item Dopplerverbreiterte Spektroskopie
  \begin{itemize}
   \item Scanne die vier Linien mit und ohne Mathe-Einheit
   \item Bestimme die Halbwertsbreiten der Linien und vereiche sie mit dem theoretischen Wert
   \item Bestimme den Abstand der Linien
   \item Bestimme den Absorbtionsquerschnitt der vier Linien
   \item Beobachte qualitativ das Fluoreszenzsignal
  \end{itemize}
 \item Dopplerfreie Spektroskopie
  \begin{itemize}
   \item Ordne die Lambdips und Crossover-Resonanzen anhand des Termschemas von Rubidium den optischen Ubergängen zu!
   \item Bestimme für beide Isotope die Hyperfeinstruktur des Grundzustandes!
   \item Miss für beide Isotope die Hyperfeinstruktur des angeregten Zustandes $5^2P_{\frac{3}{2}}$!
   \item Bestimme den Absorbtionsquerschnitt der vier Linien
   \item Beobachte qualitativ das Fluoreszenzsignal
  \end{itemize}
 \item Frequenzmodulationspektroskopie
  \begin{itemize}
   \item Frequenzmodulation des Resonators
  \begin{itemize}
   \item Vergleiche das FM-Signal mit dem DC-Signal aus der schnellen Photodiode
   \item Bestimme die Phasenverschiebung bei der das Signal invertiert erscheint
   \item Finde eine Phasenverschiebung, bei der man die Ableitung des Transmissionssignals sieht
  \end{itemize}
   \item FM-Spektroskopie an der Zelle
  \begin{itemize}
   \item Versuche, die Hyperfeinstruktur von Rubidium noch genauer auszumessen
  \end{itemize}
  \end{itemize}
  \item Messungen zur Polarisation von Licht
  \item Messungen zur Magnetooptik
\end{itemize}
\newpage

\section{Theoretische Grundlagen}
Hier soll ein Einblick in die theoretischen Grundlagen gegeben werden.
\subsection{Der Laser}
Die Abkürzung LASER steht für "Light Amplification by stimulated Emision of Radiation". Es handelt sich also um einen lichtverstärkenden Effekt, der auf stimmulierter Emission von Strahlung beruht. Trotz den zahlreichen Realisierungen ist das physikaliche Prinzip eines jeden Lasers immer das Gleiche, es beruht auf einer Besetzungsinversion und stimmulierter Emission. Desweiteren benötigt man zur erzeugung von Laserlicht einen optischen Resonator. Um die Vorgänge beim lasen zu verstehen muß man also zunächst auf das Prinzip der Besetzungsinversion eingehen.

\subsubsection{Besetzungsinversion}
Von einer Besetzungsinversion spricht man wenn ein angeregter Zustand eines Systems stärker besetzt ist ist als ein energetisch niedrig gelegener. Man will also eine signifikante Abweichung von der thermischen Besetzungsverteilung der Zustände in einem aktiven Medium erreichen. Dies ist nur mit einem physikalischen Trick technisch zu realisieren.
Im folgenden sollen $Z_g$ und $Z_a$ die Zustände invertierter Besetzung mit der Energiebeziehung $E_{Z_g}<E{Z_a}$ sein.
Man erreicht Besetzungsinversion, wenn der Grundzustand $Z_g$ schnell relaxiert und die Lebensdauer kürzer ist als die durch spontane Emission bestimmte Lebensdauer des angeregten  Zustands $Z_a$. $Z_g$ muss schnell entleert werden, damit die Besetzungsinversion erhalten bleibt. Dagegen muss $Z_a$ langlebig sein, damit die Besetzung nicht durch spontane Emission soweit reduziert wird, dass die Inversion wieder ausgeglichen wird. Um nun $Z_a$ langfristig zu befüllen wird Energie mittels optischem Pumpen in das System geführt. Wird in dieses System nun Licht der Energie $E_{Z_a}-E_{Z_g}$ eingestrahlt, so kommt es zu stimmulierter Emmission.\\
Es existieren wie erwähnt viele technische Umsetzungen des Lasers, wir verwenden einen Halbleiderdiodenlaser.
Um eine Konzentration der induzierten Emission auf wenige Moden zu erreichen braucht man desweiteren einen optischen Resonator.

\subsubsection{Optischer Resonator}
Ein optischer Resonator ist eine Anordnung von Spiegeln, die dazu dient, Licht möglichst oft hin und her zu reflektieren. Aufgrund von Interferenz bildet sich im Resonator dann eine stehende Welle, wenn die optische Weglänge des Resonators ein Vielfaches der halben Wellenlänge des eingestrahlten Lichts beträgt. In diesem Versuch kommen zwei Resonatoren zum Einsatz. Der erste befindet sich direkt hinter dem Laser, er bildet mit der Diode eine sogenannte Litrow-Anordnung.
\subsubsection{Litrow-Anordnung}
In dieser Anordnung reflektiert man einen Teil des von der Laserdiode emittierten Lichts wieder in sich zurück. Dies geschieht mit einem Refelexionsgitter und einem Kollimator, der zwischen Laserdiode und Reflexionsgitter sitzt (Abb. \ref{litrow}). Je nach Position des Reflexionsgitters wird eine andere Wellenlänge in die Laserdiode reflektiert und verstärkt. Für die verstärkte Wellenlänge erhält man in Abhängigkeit des Winkels
\begin{align}
\label{litrowpeak}
 2d sin(\alpha) = \lambda.
\end{align}
%?SKIZZE litrow
\subsubsection{Spiegel-Resonator}
%?SKIZZE cavity
Der zweite Resonator in diesem Versuch ist ein sogenannter Spiegel-Resonator. Ein Spiegel-Resonator  besteht aus zwei auf der optischen Achse liegenden,
teildurchlässigen Spiegeln $S_1$ und $S_2$ im Abstand L (Abb. \ref{cavity}). Ein Resonator heißt optisch stabil, wenn ein paraxialer Lichtstrahl im Resonator
auch nach vielen Reflexionen an den Spiegeln den Resonator nicht verlässt. Wir verwenden in diesem Versuch einen konfokalen Resonator, weil er auch bei kleinen Längenänderung stabil bleibt. Die Krümmungsradien $R_1$ und $R_2$ der beiden Spiegel entsprechen in diesem Fall genau der Länge L des Resonators, es gilt $R1 =R2 = L$. Beim konfokalen Resonator beträgt der Gangunterschied zwischen durchlaufendem und reflektierem Strahl $4L$. Es kommt zu konstruktiver Interferenz, wenn dieser Gangunterschied gerade $m\lambda$  beträgt. Daraus folgt für die Frequenzen der Transmissionslinien:
\begin{align}
\label{cavityfreq}
 f_m = \frac{mc}{4Ln}\hspace{15pt}\textnormal{ mit Brechungsindex }n\textnormal{ und } m \in \mathbb N
\end{align}
%?SKIZZE cavitypeaks
Der Abstand zweier Transmissionslinien (Abb. \ref{cavitypeaks}) wird als freier Spektralbereich ($\Delta f_{FSR}$) des Resonators bezeichnet. Mit Formel \ref{cavityfreq} erhält man
\begin{align}
\label{freespecrange}
 \Delta f_{FSR} = f_{m+1}-f_{m} = \frac{c}{4Ln}
\end{align}
 
Durch Reflexionsverluste an den Spiegeln haben die Linien eine endliche Halbwertsbreite $\delta f$. Ein Maß für die Anzahl der Umläufe eines Photons im Resonator stellt die Finesse $F$ des Resonators dar. Sie hängt direkt von der Reflektivität der Spiegel ab und berechnet sich zu Quotient aus freiem Spektralbereich und Halbwertsbreite
\begin{align}
\label{finesse}
 \Delta F  = \frac{\Delta f_{FSR}}{\delta f}
\end{align}
\subsection{Spektroskopie}
In unserem Versuch sollen die $D2$-Linieen von zwei Rubidiumisotopen untersucht werden. Rubidium (Rb) ist ein Alkalimetall, d.h. es besitzt ein einzelnes Elektron in seiner Außenschale und hat somit ein wasserstoffähnliches Termschema. Es existieren zwei Isotope ($^{85}Rb$ und $^{87}Rb$) und es kommt in der Natur nicht ungebunden vor. $^{85}Rb$ ist ein stabiles Isotop, $^{87}Rb$ zefällt sehr langsam.

\subsubsection{Feinstrukturaufspaltung}

\subsubsection{Hyperfeinstrukturaufspaltung}
Die Hyperfeinstrukturaufspaltung resultiert aus der Kopplung des kernmagnetischen Moments $\mu_I$ mit dem am Kernort durch die Hüllenelektronen erzeugten Magnetfeld $B_J$.
Durch die Kopplung ist der Gesamtdrehimpuls $\vec{F}$, der die Summe aus Hüllendrehimpuls $\vec{J}$ und Kernspin $\vec{I}$ ist, gequantelt:
\begin{align*}
 \left|\vec{F}\right| = \hbar \sqrt{F(F+1)}
\end{align*}
Die Quantenzahl $F$ nimmt in ganzzahligen Schritten die Werte von $-I$ bis $I$ an.
\subsubsection{Linienverbreiterung}
\begin{itemize}
 \item \textbf{natürliche Linienbreite} \\
    Wegen der Heisenbergschen Unschärferelation für Energie $E$ und Zeit $t$:
    \begin{align*}
    \Delta E \Delta t \geq \hbar
    \end{align*}
    verursacht die endliche Lebensdauer $\tau$ angeregter Zustände eine natürliche Linienbreite $\Delta f_N$.
    \begin{align}
    \label{natbreite}
    \Delta f_N  = \frac{\Delta E}{h} = \frac{1}{2 \pi \tau}
    \end{align}
    Der Zustand $5^2P_{\frac{3}{2}}$ hat (bei beiden Isotopen des verwendeten Rubidiums) eine Lebensdauer von $\tau =26,5 ns$
    Die natürliche Linienbreite der Rb-Linien beträgt demnach $\Delta f_N \approx 6 Mhz$
 \item \textbf{Dopplerverbreiterung} \\
    Die Atome in der Rubidiumzelle befinden sich in einem thermischen Gleichgewicht, sie bewegen sich also. Daher erscheinen die Spektrallinien dopplerverbreitert. Es lässt sich leicht zeigen das für die vollen Halbwertsbreiten der Linien $\Delta f_{FWHM}$dann gilt:
    \begin{align}
    \label{FWHM}
     \Delta f_{FWHM} = \frac{1}{\lambda} \sqrt{\frac{8 k_B T ln 2}{m}}
    \end{align}
    Dabei ist $k_B$ die Boltzmannkonstante, $m$ die Masse und $\lambda$ die Ubergangswellenlänge.
Für Zimmertemperatur und Rubuidium erwartet man also eine Halbwertsbreite von $Delta f_{FWHM} \approx 0,5 Ghz$
\end{itemize}

\subsubsection{Absorbtionsquerschnitt}

\subsubsection{Dopplerverbreiterte Spektroskopie}
Wie der Name schon verrät spektroskopiert man in der dopplerverbreiterterten Spektroskopie Linien von Übergängen welche
dopplerverbreitert sind. In unserem Versuch wird hier die Transmission der Linien von Rubidium beobachtet. Durch die große
Linienbreite lässt sich mit Hilfe der dopplerverbreiterten Spektroskopie  bei Rubidium nur die Feinstruktur und nicht die
Hyperfeinstruktur auflösen. Im Wesentlichen wird bei dieser Spektroskopie Licht auf das zu untersuchende Medium gestrahlt und hiter dem Medium die Intensität gemessen. Durch Variation der Wellenlänge des eingestrahlten Lichts lassen sich verschiedene Zustände im Medium anregen: Entspricht die eingestrahlte Wellenlänge gerade der eines Übgangs im Medium, so wird dieser Übergang angeregt und somit das einfallende Licht absorbiert. Dies bedeutet wiederum, dass die hinter dem Medium gemessene Intensität abnimmt. Dieses Verfahren nennt man auch transmissions Spektroskopie.

\subsubsection{Dopplerfreie Spektroskopie}
Durch eine geschickte anordnung des Versuchsaufbaus lässt Spektroskopie auch dopplerfrei durchführen.
Hierzu lässt man das zu Spektroskopierende Medium von zwei Strahlen durchlaufen welche sich im Medium überlappen aber in entgegengesetzter Richtung verlaufen. Oft wird der Strahl nach dem ersten Durchlauf einfach wieder in sich zurück reflektiert. Bei dieser Anordnung gibt es zwei mögliche Situationen:
\begin{itemize}
 \item Die Wellenlänge des eingestrahlten Lichts entspricht gerade einem Übergang der Atome die ruhen.
 \item Die Wellenlänge entspricht gerade nicht dem Übergang der ruhenden Atome.
\end{itemize}
Im ersten Fall werden durch den ersten Strahl, auch Sättigungsstrahl genannt, gerade die Atome in das höhere Nieveau des Übergangs angeregt welche ruhen. Für den rücklaufenden, zweiten Strahl, auch Abfragestrahl genannt, erscheint das Medium transparent, da sich bei ausreichender Intensität des Sättigungsstrahl die meisten verfügbaren ruhenden Atome schon im angeregten Zustand befinden. Im Zweiten fall dagegen, besteht die Übereinstimmung von Wellenlänge mit Übergang der einen Strahlrichtung gerade aus der entgegengesetzten Geschwindigkeitsklasse der anderen Strahlrichtung. Somit ist in diesem Fall keine tranzsparenz des Mediums für den Rücklaufenden Strahl gegeben. Die bei dieser Anordnung beobachteten Peaks nennt man \textbf{Lambdips}. Hat das Medium mehrere Übergänge und ist deren Abstand kleiner als die maximal auftretende Dopplerverschiebung, so entstehen bei der dopplerfreien Spektroskopie auch noch weitere Peaks, die sogenannten \textbf{crossover Resonanzen}. Diese liegen immer genau zwischen zwei Lambdips. Liegt die eingestrahlte Wellenlänge genau zwischen zwei Übergängen, so existiert eine Geschwindigkeitsklasse die für den einen Übergang gerade dem Sättigungsstrahl entspricht und für den anderen Übergang dem Abfragestrahl. Da der Sättigungsstrahl diese Geschwindigkeitsklasse somit in den einen angeregten Zustand anhebt, erscheint das Medium für den Abfragestrahl (dieser würde ja gerade die selbe Geschwindigkeitsklasse, also die selben Atome in den anderen Zustand anheben) transparent.

\section{Versuchsaufbau}
Der Versuchsaufbau besteht aus diversen Optischen und Elektronischen Komponenten die im folgenden Erläutert werden.
Die Beschreibung der konkreten Verwendung für die einzelnen Versuchsteile erfolgt mit in der Auswertung.
\subsection{Optik}
Grundsätzlich werden alle optischen Elemente auf eine Metallplatte mit einem Lochraster geschraubt. Hierdurch kann ein ungewolltes verschieben der Elemente verhindert werden. Auf dieser Platte befindet sich bereits der Laser zusammen mit seinem Resonator, außerdem ein Strahlheber welcher den Laserstrahl auf eine angenehme Arbeitshöhe hebt. Zudem befinden sich ebenfalls ein Prismenpaar, welches den Strahl möglichst Kreisförmig machen soll, und ein optischer Isolator (lässt den Strahl nur in eine Richtung durch)
\subsection{Elektronik}
\section{Auswertung}
\subsection{Der Laser}
Zunächst untersuchten wir die Laserschwelle grob mit der Infrarotkarte. Wir sahen ab $I_{Laser} = () mA$ ein einsetzen der Strahlung.
Die Photodiodenkennlinien mit den drei Wiederständen sind in Abb. \ref{pdkennlinien} dargestellt. Wir bestimmten daraus die jeweiligen Laserschwellen zu:
\begin{align*}
 I_{Laser, \Omega} = () mA \\
 I_{Laser, \Omega} = () mA \\
 I_{Laser, \Omega} = () mA
\end{align*}
hierbei handelt es sich um geschätzte Ablesefehler.
\begin{figure}
 \includegraphics[width=0.9\linewidth]{pictures/pdkennlinien.eps}
 \caption{Kennlinien der passiven Photodiode für die drei schaltbaren Wiederstände}
 \label{pdkennlinien}
\end{figure}

\subsection{Der Resonator}
Um den freien Spektralbereich bestimmen zu können braucht man zunächst eine Frequenzeichung. Wir modulierten also unterschiedliche Frequenzen auf die Laserfrequenz.Für die Abstände der Seitenbänder $d_{\mu s}$ und der Modulationsfrequenz $f_M$muss dann folgender Zusammenhang gelten:
\begin{align*}
 $f_M$ = A * d_{\mu s}
\end{align*} 
in Abb. \ref{eichung} sind unsere Daten Zusammen mit dem Fit dargestellt. Die Fehler sind auch hier Ablesefeher. Wir erhielten also für $A$:
\begin{align*}
 A = ( \pm ) \frac{1}{s^2}
\end{align*} 
Wir bestimmten den freien Spektralbereich $\Delta f_{FSR}$ indem wir die Abstände $d_i$ von zwölf Resonatorpeaks mittelten. Auch hier schätzten wir während der Messung einen Fehler auf die Ablesegenauikeiten unserer Messdaten. Aus obiger Energieeichung erhielten wir dann:
\begin{align*}
 \Delta f_{FSR} = (\pm) Ghz
\end{align*}
hierbei berechnete sich der Fehler nach der Gaußschen Fehlerfortpflanzung zu
\begin{align*}
 s_{\Delta f_{FSR}} = \Delta f_{FSR} * \sqrt{ \left( \frac{s_A}{A} \right)^2 + \left( \frac{s_{d_i}}{d_i} \right)^2}
\end{align*}

nun lässt sich über $\Delta f_{FSR} = \frac{c}{4Ln}$ auch die Länge des Resonators bestimmen. Wir erhielten
\begin{align*}
 s_{\Delta f_{FSR}} = \Delta f_{FSR} * \sqrt{ \left( \frac{s_A}{A} \right)^2 + \left( \frac{s_{d_i}}{d_i} \right)^2}
\end{align*}
dabei überträgt sich hier der relative Fehler von $\Delta f_{FSR}$.  \\

Weiter bestimmten wir mit dem Osziloskop die Halbwertsbreiten $\delta f$ von fünf Resonatorpeaks und mittelten diese, um damit die Finesse $F$ des Resonators zu bestimmen. Wir erhielten 
\begin{align*}
 F = \frac{\Delta f_{FSR}}{\delta f} = ( \ pm )
\end{align*}
hierbei berechnete sich der Fehler nach der Gaußschen Fehlerfortpflanzung zu
\begin{align*}
 s_F = F * \sqrt{ \left( \frac{s_{\Delta f_{FSR}}}{\Delta f_{FSR}} \right)^2 + \left( \frac{s_{\delta f}}{\delta f} \right)^2}
\end{align*}
Leider mussten wir nach beendigung des Versuchs feststellen das wir vergessen hatten die Piezo-Frequenz-Charakteristik zu vermessen. 
Wir charakterisierten die Frequenzmodulation und verglichen sie mit den theoretischen Werten, die Ergebnisse hiervon sind in Abb. \ref{bessel} zu sehen.
%?SKIZZE bessel
\subsection{Dopplerverbreiterte Spektroskopie}
\subsubsection{Linienbreite}
Wir untersuchten die vier Linien mit und ohne Mathe-Einheit, als Differenz und Quotient. Aus der Linienform der Referenzdiode und den Ergebnissen beim fitten schlossen wir, das ein eine Faltung aus vier Gaußkurven mit einem Polynom 2.Grades die Daten am besten repräsentiere. Das Ergebniss unseres Fits an ein Signal, gemessen als Quotitnt, ist in Abb \ref{doppler} zu sehen. Aus den Parametern der Gaußbreiten des Fits berechneten wir Die volle Halbwertsbreite in $\mu s$. Über einen mitaufgenommenen Frequenzkamm rechneten wir sie in eine Frequenzen um. Zuletzt bildeten wir das gewichtete Ḿittel und erhielten als Dopplerbreite $\Delta f_{FWHM}$:
\begin{align*}
 \Delta f_{FWHM} = (\pm) Ghz 
\end{align*}
hierbei berechnete sich der Fehler nach der Gaußschen Fehlerfortpflanzung
\subsubsection{Abstände der Linien}
Aus den Fits in Abb \ref{doppler} erhielten wir die Schwerpunkte der Peaks. Wir rechneten Sie wieder in eine Frequenz um und ordneten sie dann ihren Übergängen zu. Das Ergebniss ist in Tabelle \ref{dist} dargestellt.
\begin{center}
\begin{tabular}{|l|lll|}
\label{dist}
\hline
Übergang & × & × & ×\\
Abstand [Ghz] & × & × & ×
\hline
\end{tabular}
\end{center}
\subsubsection{Absobtionsquerschnitt}
Da der Absorbtionsquerschnitt direkt vom Quotient der Intensitäten vor und hinter der Zelle abhängt, konnten wir diesen mittels des Gaußparameters bestimmen.
Wir erhielten
\begin{align*}
 \sigma = \frac{ln \left( \frac{I_0}{I_d} \right) }{nd}
\end{align*}
\section{Zusammenfassung}
\section{Anhang}


\subsection{Quelltext}

%\lstinputlisting[language=python, extendedchars=false, inputencoding=utf8]{../staebe/staebe.py}

\end{document}
