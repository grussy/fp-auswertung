\documentclass[12pt]{article}
\usepackage[paper=a4paper,left=30mm,right=30mm,top=35mm,bottom =35mm]{geometry}
\usepackage[utf8]{inputenc}
\usepackage[T1]{fontenc}
\usepackage{stmaryrd}
\usepackage{setspace}
\usepackage{mathrsfs}
\usepackage[ngerman]{babel}
\usepackage{amssymb}
\usepackage{amsmath}
\usepackage{fancyhdr}
\usepackage[dvips,unicode,colorlinks,linkcolor=black]{hyperref} 
\usepackage{graphicx}
\usepackage{float}
\usepackage{listings}

\pagestyle{fancy}
\lfoot{}
\rfoot{Paul Kremser, Tobias Grussenmeyer}
\cfoot{\thepage}
\fancyhead[L]{FPII Versuch: Laserspektroskopie}
\renewcommand{\headrulewidth}{0.6pt}
\renewcommand{\footrulewidth}{0.6pt}
\setlength{\headheight}{30pt}
\setlength{\parindent}{0pt}
% Für die Wahl der Schriftart
\newcommand{\changefont}[3]{
\fontfamily{#1} \fontseries{#2} \fontshape{#3} \selectfont}
%\renewcommand{\vec}[1]{\mathbf{#1}}
\renewcommand*\vec[1]{{\mbox{\boldmath\ensuremath{#1}}}}

\lstset{
	basicstyle=\ttfamily\scriptsize\mdseries,
	keywordstyle=\bfseries\color{black},
	identifierstyle=,
	commentstyle=\color{black},	
	stringstyle=\itshape\color{black},
	numbers=left,
	numberstyle=\tiny,
	stepnumber=1,
	breaklines=true,
	frame=none,
	showstringspaces=false,
	tabsize=4,
	backgroundcolor=\color{white},
	captionpos=b,
	float=htbp,
	frame=tlrb
%	extendedchars=true
}

\begin{document}
% keine Hurenkinder und Schusterjungen
\clubpenalty = 10000
\widowpenalty = 10000 
\displaywidowpenalty = 10000

\onehalfspacing
% Schriftart
\changefont{ptm}{m}{n} 

\begin{titlepage}
\author{Paul Kremser, Tobias Grussenmeyer}
\title{Versuch: Laserspektroskopie}
\date{Versuchsdurchführung: 30. März bis 2.April 2010} 
\maketitle
\thispagestyle{empty}
\end{titlepage}
%Leider schafften wir nur einen geringen Teil der gestellten Aufgaben. Ein wichtiger Teil unserer Messdataen, der des 
%letzten Tages unserer Messungen, ging verloren. Aufgrund eines Feiertages hatten wir desweiteren nur vier von fünf Möglichen Tagen zur Verfügung.

\tableofcontents
\thispagestyle{empty}
\newpage
\pagenumbering{arabic}
\section{Überblick}
In diesem Versuch soll das Prinzip der Laserspektroskopie verstanden und an einem Beispiel angewand werden. Zusätzlich sollen Messungen zur Polarisation von
Licht und Magnetooptik gemacht werden.
Unter Laserspektroskopie versteht man verschiedene Verfahren zur Spektroskopie, in denen Laser zur Untersuchung von atomaren oder molekularen Spektren eingesetzt werden. 
In unserem Versuch wird das zu untersuchende Atom ($^{85}Rb$ und $^{87}Rb$) durch Lasereinstrahlung von einem unteren auf ein höheres Energienievau gehoben.
Dadurch kommt es beim variieren der Laserfrequenz zu Absorbtionsmaxima die mit einem Detektor gemessen werden können. 
\section{Aufgabenstellung}
\begin{itemize}
 \item Der Laser
  \begin{itemize}
   \item Bestimme die Laserschwelle 
   \item Nimm für drei Wiederstände eine Spannungskurve auf
  \end{itemize}
 \item Resonator und Frequenzmodulation
  \begin{itemize}
   \item Bestimme den freien Spektralbereich $\Delta f_{FSR}$ des Resonators
   \item Aus $\Delta f_{FSR}$ die Länge des Resonators
   \item Die Finesse F des Resonators
   \item Miss die Strom-Frequenz Charakteristik der Laserdiode
   \item Charakterisiere die Frequenzmodulation
  \end{itemize}
 \item Dopplerverbreiterte Spektroskopie
  \begin{itemize}
   \item Scanne die vier Linien mit und ohne Mathe-Einheit
   \item Bestimme die Halbwertsbreiten der Linien und vereiche sie mit dem theoretischen Wert
   \item Bestimme den Abstand der Linien
   \item Bestimme den Absorbtionsquerschnitt der vier Linien
   \item Beobachte qualitativ das Fluoreszenzsignal
  \end{itemize}
 \item Dopplerfreie Spektroskopie
  \begin{itemize}
   \item Ordne die Lambdips und Crossover-Resonanzen anhand des Termschemas von Rubidium den optischen Ubergängen zu!
   \item Bestimme für beide Isotope die Hyperfeinstruktur des Grundzustandes!
   \item Miss für beide Isotope die Hyperfeinstruktur des angeregten Zustandes $5^2P_{\frac{3}{2}}$!
   \item Bestimme den Absorbtionsquerschnitt der vier Linien
   \item Beobachte qualitativ das Fluoreszenzsignal
  \end{itemize}
 \item Frequenzmodulationspektroskopie
  \begin{itemize}
   \item Frequenzmodulation des Resonators
  \begin{itemize}
   \item Vergleiche das FM-Signal mit dem DC-Signal aus der schnellen Photodiode
   \item Bestimme die Phasenverschiebung bei der das Signal invertiert erscheint
   \item Finde eine Phasenverschiebung, bei der man die Ableitung des Transmissionssignals sieht
  \end{itemize}
   \item FM-Spektroskopie an der Zelle
  \begin{itemize}
   \item Versuche, die Hyperfeinstruktur von Rubidium noch genauer auszumessen
  \end{itemize}
  \end{itemize}
  
\end{itemize}

\section{Theoretische Grundlagen}
\section{Durchführung}
\section{Auswertung}
\section{Zusammenfassung}

\section{Anhang}

\subsection{Quelltext}

%\lstinputlisting[language=python, extendedchars=false, inputencoding=utf8]{../staebe/staebe.py}

\end{document}
